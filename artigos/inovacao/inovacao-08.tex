\documentclass[article,12pt,onesidea,4paper,english,brazil]{abntex2}

\usepackage{lmodern, indentfirst, nomencl, color, graphicx, microtype, lipsum,textcomp}			
\usepackage[T1]{fontenc}		
\usepackage[utf8]{inputenc}		

\setlrmarginsandblock{2cm}{2cm}{*}
\setulmarginsandblock{2cm}{2cm}{*}
\checkandfixthelayout

\setlength{\parindent}{1.3cm}
\setlength{\parskip}{0.2cm}

\SingleSpacing

\begin{document}
	
	\selectlanguage{brazil}
	
	\frenchspacing 
	
	\begin{center}
		\LARGE DESENVOLVIMENTO DE UM SISTEMA DE INFORMAÇÃO PARA O
		GERENCIAMENTO E CONTROLE DO BANCO DE GERMOPLASMA DO BIOTÉRIO DO IPEN, COMO AUXÍLIO À PESQUISA DE NOVOS RADIOFÁRMACOS\footnote{Trabalho realizado dentro da área de Conhecimento CNPq: Ciência da Computação (1030007) com financiamento do instituto Federal de Rondônia - IFRO.}
		
		\normalsize
		Clayton Ferraz Andrade\footnote{Orientador, clayton.andrade@ifro.du.br, \textit{Campus} Ji-paraná.} 
		Roberto Simplício Guimarães\footnote{Colaborador(a), roberto.simplicio@ifro.edu.br, \textit{Campus} Vilhena.} 
	\end{center}
	
	\noindent Os radiofármacos são fármacos radioativos utilizados no diagnóstico ou tratamento
	de patologias e disfunções do organismo humano. Vários radioisótopos são
	utilizados na preparação de radiofármacos, entre os quais o tecnécio-99m (99mTc),
	que apresenta características físicas ideais para aplicação em Medicina Nuclear
	Diagnóstica. Para que um radiofármaco tenha seu uso permitido, assim como os
	fármacos, a Agência Nacional de Vigilância Sanitária, versa que é necessário:
	“relatório de ensaios pré-clínicos: toxicidade aguda, subaguda e crônica, toxicidade
	reprodutiva, atividade mutagênica, potencial oncogênico de acordo com a legislação
	específica”, esses resultados pré-clínicos consiste em uso de modelo animal. Uma
	vez que manter diversas linhagens em um biotério é, muitas vezes, um desejo
	justificável da comunidade científica, uma vez que permite o desenvolvimento de
	vários ensaios experimentais. Entretanto, isso é muito caro e exige a adoção de
	diversas outras providências, tais como capacitação de recursos humanos, além da
	adequação dos espaços físicos (algumas vezes por exigência técnica da própria
	linhagem) e dos insumos básicos para a manutenção dos animais.
	Já por sua vez, os embriões do banco de germoplasma, não possuem tamanhas
	necessidades, bastando lhes um suprimento de nitrogênio líquido para que
	permaneçam congelados. Dado a problemática, o trabalho tem por objetivo o
	desenvolvimento de um sistema de informação, através de um software, para o
	gerenciamento e controle do banco de criopreservação de germoplasma do biotério
	do IPEN-USP visando auxiliar na pesquisa de novos fármacos e radiofármacos.
	Assim, o estabelecimento do banco permite ao biotério a manutenção de um grande
	número de linhagens a baixo custo. Espera-se que uma vez que o software
	implantado gerencie todas essas etapas, mantendo um rígido controle sobre o
	processo e ao mesmo tempo oferecesse relatórios estratégicos e nível gerencial e
	acesso on-line a comunidade usuária desse banco, haverá : 1) controle centralizado
	das ações, 2) uniformidade no processo entre as entidades que integram uma rede,
	3) rapidez na recuperação das informações para a gestão do centro, 4) atendimento
	aos princípios do programa 3Rs, pois os número de animais seria reduzido ao
	mínimo possível para testes in vivo. O presente trabalho está em fase de testes,
	sendo testado em pareceria com o CEMIB-UNICAMP, centro de bioterismo com
	certificação internacional e referência na América Latina. Os resultados finais
	deverão ser apresentamos no 1° semestre de 2018.
	
	\vspace{\onelineskip}
	
	\noindent
	\textbf{Palavras-chave}: Software. Radiofármacos. Criopreservação.
	
\end{document}