\documentclass[article,12pt,onesidea,4paper,english,brazil]{abntex2}

\usepackage{lmodern, indentfirst, nomencl, color, graphicx, microtype, lipsum}			
\usepackage[T1]{fontenc}		
\usepackage[utf8]{inputenc}		

\setlrmarginsandblock{2cm}{2cm}{*}
\setulmarginsandblock{2cm}{2cm}{*}
\checkandfixthelayout

\setlength{\parindent}{1.3cm}
\setlength{\parskip}{0.2cm}

\SingleSpacing

\begin{document}
	
	\selectlanguage{brazil}
	
	\frenchspacing 
	
	\begin{center}
		\LARGE FREEZER INTELIGENTE\footnote{Área de Conhecimento CAPES/CNPq: Ciência da Computação (10300007) / Hardware (10304010).}
		
		\normalsize
		João Roberto Bond da Silva\footnote{Bolsista, jobond46@gmail.com, Discente de Tecnologia em Análise e Desenvolvimento de Sistemas, Campus Vilhena.} 
		Roberto Simplício Guimarães\footnote{Coordenador, roberto.simplicio@ifro.edu.br, Professor do curso de Tecnologia em Análise e Desenvolvimento de Sistemas, Campus Vilhena}
	\end{center}
	
	\noindent É comum ver em noticiários que milhares de doses de vacinas foram jogadas no lixo
	por falta de refrigeração, com isso, dinheiro público sendo desperdiçado. Tendo em
	vista essa problemática, faz-se necessário a criação de uma central de
	monitoramento da refrigeração dessas doses por um preço acessível. O Freezer
	Inteligente é um conjunto de sensores e atuadores controlados por um
	microcontrolador Atmel (Arduino Mega) que visam monitorar e informar através de
	torpedos SMS a temperatura, umidade, integridade e diversas outras características
	dos refrigeradores, com o objetivo de diminuir a perda de produtos refrigerados. Foi
	usado um artigo de alunos da UTFPR (Universidade Tecnológica do Paraná) que
	explicam sobre a perda das doses de vacina (Cadernos Saúde Coletiva (2013) e o
	livro “Primeiros Passos com o Arduino” sobre sistemas embarcados (Massimo Banzi,
	2013). Para a criação do produto, foi usada a ideia de entrega contínua com
	algumas adaptações. Neste método, uma funcionalidade é criada e deixamos o
	código-fonte pronto para começar outra etapa. Isto é feito até que o projeto chegue
	ao final e possamos implantar no refrigerador do cliente. Este método é muito
	utilizado em empresas de desenvolvimento de softwares, tendo em vista que o
	mesmo diminui o tempo ocioso no projeto. O projeto ainda não chegou ao final,
	porém, um resultado parcial já foi obtido. Foi visto através de uma pesquisa simples
	que seria interessante uma plataforma web onde mostra através de gráficos e
	planilhas para o cliente os dados sobre o seu freezer e não apenas mensagens de
	aviso pelo telefone.O código que faz a medição de temperatura, umidade e
	eletricidade do refrigerador já foi desenvolvido. O envio de torpedos SMS está sendo
	implementado e o painel de monitoramento web é o próximo passo a ser dado pela
	equipe. A importância e a aplicação deste produto para o mercado de refrigeradores
	é gigantesca, principalmente o custo e benefício do produto. Uma central de
	monitoramento hoje custa em média de dois à três mil reais no Brasil e o Freezer
	Inteligente traz as mesmas funcionalidade com algumas particularidades que estas
	centrais por cerca de quinhentos reais.
	
	\vspace{\onelineskip}
	
	\noindent
	\textbf{Palavras-chave}: Automação. Arduino. Inovação.
	
\end{document}