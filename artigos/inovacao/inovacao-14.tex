\documentclass[article,12pt,onesidea,4paper,english,brazil]{abntex2}

\usepackage{lmodern, indentfirst, nomencl, color, graphicx, microtype, lipsum}			
\usepackage[T1]{fontenc}		
\usepackage[utf8]{inputenc}		

\setlrmarginsandblock{2cm}{2cm}{*}
\setulmarginsandblock{2cm}{2cm}{*}
\checkandfixthelayout

\setlength{\parindent}{1.3cm}
\setlength{\parskip}{0.2cm}

\SingleSpacing

\begin{document}
	
	\selectlanguage{brazil}
	
	\frenchspacing 
	
	\begin{center}
		\LARGE PROTÓTIPO DIDÁTICO DE CONVERSOR COMPUTACIONAL DE NÚMEROS PARA O ENSINO DE INFORMÁTICA\footnote{Trabalho realizado dentro da área de Ciência da Computação.}
		
		\normalsize
		Juliano de Mello\footnote{Bolsista (Ensino Médio Integrado ao Técnico), juliano.ifro@gmail.com, Campus Vilhena.} 
		Eberson Taynan Tomazelli\footnote{Bolsista (Ensino Médio Integrado ao Técnico), eberson\_taynan@hotmail.com, Campus Vilhena.} 
		Marco A. A. Andrade\footnote{Colaborador(a), marco.andrade@ifro.edu.br, Campus Vilhena.} 
		Cleyton Pereira dos Santos\footnote{Orientador(a),cleyton.santos@ifro.edu.br, Campus Vilhena.}
		Rodrigo Alécio Stiz\footnote{Co-orientador(a), rodrigo.stiz@ifro.edu.br, Campus Vilhena.} 
	\end{center}
	
	\noindent A proposta do protótipo didático de um conversor computacional de números para o
	ensino de informática decorreu das dificuldades constatadas no ensino de alguns
	conteúdos da área de informática, como fundamentos de informática, lógica de
	programação, bits e bytes, entre outros inseridos em diversas disciplinas. Os
	referidos conteúdos são primordialmente teóricos, com abstrações distantes do
	quotidiano, e em consequência disso passa a existir uma tendência muito forte do
	aluno não dar importância ao seus estudos e aprendizados, perdendo com isso
	valiosos conhecimentos e fundamentais para sua futura profissão. Assim, a proposta
	do conversor didático de números é de pretender ser um recurso a mais no processo
	de ensino/aprendizagem, o qual docente faça uso para despertar, estimular e
	facilitar o aluno a ter um aprendizado significativo dos conteúdos ora difíceis de
	serem aprendidos. Por conseguinte, o protótipo também pretende ser um recurso a
	ser expandido e estar correlacionado a duas correntes de pensamento sobre o
	processo de ensino/aprendizagem, a da Aprendizagem Significativa de Ausubel e o
	Construtivismo de Piaget, tornando o dispositivo proposto um recurso que desperta
	no aluno a curiosidade e o interesse, e facilite o processo de mediação da
	aprendizagem pelo professor com o uso de algo concreto para o aluno manipular e
	interagir. Portanto, o conversor computacional de números de bases matemáticas
	diferentes é um dispositivo eletrônico para facilitar o ensino de conteúdos teóricos de
	informática, como bits, bytes, conversão números de bases binárias para decimais, hardware, software, lógica de programação e algoritmos. Para construção do
	protótipo do conversor foram realizadas pela equipe, pesquisas bibliográficas, documentais e em meio eletrônico, e em seguida foi elaborado uma versão virtual do
	protótipo via programa de simulação, e com isso foi projetado o modelo do esquema
	eletrônico do conversor. Posteriormente o esquema do protótipo foi construído em
	laboratório eletrônico usando protoboards, componentes eletrônicos, um Arduíno e
	um computador. Pela equipe também foram feitas tentativas de usar outras
	plataformas de embarcados, como PIC e Banana PI, porém como exigiam muito mais tempo para serem desenvolvidas foram abandonadas para em outro momento, com mais tempo, sejam retomadas no projeto.
	
	\vspace{\onelineskip}
	
	\noindent
	\textbf{Palavras-chave}: Conversor. Didático. Números.
	
\end{document}