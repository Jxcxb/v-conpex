\documentclass[article,12pt,onesidea,4paper,english,brazil]{abntex2}

\usepackage{lmodern, indentfirst, nomencl, color, graphicx, microtype, lipsum}			
\usepackage[T1]{fontenc}		
\usepackage[utf8]{inputenc}		

\setlrmarginsandblock{2cm}{2cm}{*}
\setulmarginsandblock{2cm}{2cm}{*}
\checkandfixthelayout

\setlength{\parindent}{1.3cm}
\setlength{\parskip}{0.2cm}

\SingleSpacing

\begin{document}
	
	\selectlanguage{brazil}
	
	\frenchspacing 
	
	\begin{center}
		\LARGE DECODIFICANDO SINAIS DE SATÉLITES METEOROLÓGICOS COM APARELHOS ALTERNATIVOS\footnote{Trabalho realizado dentro da Ciências Exatas e da Terra com financiamento da PROPESP – IFRO.}
		
		\normalsize
		José Valdir de Lima\footnote{Autor, joselima275@gmail.com, \textit{Campus} Ji-Paraná.} 
		Andréia Mendonça dos Santos Lima\footnote{Co-Autor, andreiamendonsa@ifro.edu.br, \textit{Campus} Ji-Paraná.} 
		Emerson Andrade de Souza\footnote{Co-Autor, emeson.ambiental@gmail.com, \textit{Campus} Ji-Paraná.} 
		Gleison Guardia\footnote{Co-Autor, gleison.guardia@ifro.edu.br, \textit{Campus} Ji-Paraná.}
		Michel Silva\footnote{Co-Autor, michel.silva@ifro.edu.br, \textit{Campus} Ji-Paraná.} 
	\end{center}
	
	\noindent Os Centros Meteorológicos do mundo, valem-se das informações que são enviadas
	a eles pelos inúmeros satélites meteorológicos que orbitam o planeta. Esta
	transmissão é livre e tem acesso liberado a qualquer pessoa, que queira fazer uso
	de suas informações, para tanto é necessário que tenham os equipamentos
	adequados para este fim. Construir ou adquirir este tipo de equipamento, pode se
	tornar um fator crucial para o estudo desse tipo de informação. A construção de
	materiais alternativos que possam complementar falhas na aquisição de um
	material, tem-se apresentado como uma direção lógica para as sociedades
	modernas, incumbindo assim responsabilidades aos centros de estudos, pesquisas
	e tecnologia nos seus desenvolvimentos. Esse trabalho teve como objetivo construir
	uma Antena QFH (quadrifiliar helixcoidal), com materiais alternativos para recepção
	de imagem de satélite. Na sequencia desse pensamento, o Instituto Federal de
	Educação, Ciência e Tecnologia de Rondônia IFRO, no seu Campus de Ji-Paraná,
	assumiu este papel social, desenvolvendo e construindo essa tecnologia, barata,
	acessível, para difundir e alavancar os estudos climáticos, aeroespaciais e
	tecnológicos da comunicação espacial. A proposta que foi apresentada por este
	projeto, promoveu o desenvolvimento de uma antena de baixo custo, um receptor
	alternativo e a integração de 3 \textit{softwares} (Orbitron, Sdrsharp e Wxtolmg) gratuitos e
	disponíveis na internet para o trabalho de trato dos dados e apresentações visuais
	em 3d das informações sobre a superfície do continente Sul-americano.. A melhoria
	no processo de construção da antena deu-se a partir do uso de tubo de PVC de
	75mm de diâmetro, utilizando-se ainda de varetas de ferro cobreado para a
	confecção QFH/PM e conectores sindal, diminuindo o custo de produção e
	aumentando a simetria da antena, com consequente melhoria na qualidade da
	captura das imagens de satélites.
	
	\vspace{\onelineskip}
	
	\noindent
	\textbf{Palavras-chave}: Antena. Meteorológico. Satélite.
	
\end{document}