\documentclass[article,12pt,onesidea,4paper,english,brazil]{abntex2}

\usepackage{lmodern, indentfirst, nomencl, color, graphicx, microtype, lipsum, textcomp}			
\usepackage[T1]{fontenc}		
\usepackage[utf8]{inputenc}		

\setlrmarginsandblock{2cm}{2cm}{*}
\setulmarginsandblock{2cm}{2cm}{*}
\checkandfixthelayout

\setlength{\parindent}{1.3cm}
\setlength{\parskip}{0.2cm}

\SingleSpacing

\begin{document}
	
	\selectlanguage{brazil}
	
	\frenchspacing 
	
	\begin{center}
		\LARGE BEBIDA LÁCTEA FERMENTADA DA AMAZÔNIA1\footnote{Trabalho realizado dentro da área de Conhecimento CNPq: Tecnologia de Alimentos com financiamento do IFRO.}
		
		\normalsize
		Camila Souza Reis\footnote{Bolsista (Discente do Curso de Engenharia Agronômica), email reiscamilasouza@gmail.com , \textit{Campus} Colorado do Oeste.} 
		Luiz Otávio Martins de Lazari\footnote{Bolsista (Discente do Curso Técnico em Agropecuária), email losnitram@gmail.com , \textit{Campus} Colorado do Oeste.} 
		Letícia Carolina Vieira\footnote{4Colaboradora (Discente do Curso de Engenharia Agronômica), email letícia\_carolina\_@gmail.com , \textit{Campus} Colorado do Oeste.} 
		Marcio Adolfo de Almeida\footnote{Orientador (Docente), email marcio.almeida@ifro.edu.br , \textit{Campus} Colorado do Oeste.} 
	\end{center}
	
	\noindent A bebida láctea se define como produto fermentado resultante da ação de
	microrganismos específicos ou de outros produtos lácteos fermentados, onde não
	pode ser submetida a tratamento térmico após a fermentação. A utilização de soro
	do leite na elaboração de bebida láctea se constitui no aproveitamento deste produto
	secundário por apresentar excelente valor nutritivo. O projeto teve como objetivo
	verificar o processo de elaboração da bebida láctea sabores da Amazônia, observar
	seu potencial de consumo e aceitabilidade, testar sua intenção de compra quanto
	aos atributos cor, textura, consistência, odor, sabor, aparência e aspecto global, e
	analisar por meio de entrevista se as pessoas possuem conhecimentos quanto aos
	derivados de soro do leite. A bebida fermentada foi elaborada na agroindústria do
	IFRO - \textit{Campus} Colorado do Oeste seguindo o fluxograma da pasteurização do
	soro, tratamento térmico a 90°C e realização da precipitação ácida com limão e ou
	cupuaçu por 5 minutos. Após esta etapa a bebida foi submetida à fermentação lática
	através da utilização de culturas termofílicas a 42°C por 8 horas em estufa de
	incubação bacteriana. Em seguida as bebidas foram adicionadas de sabor como
	cacau, limão cupuaçu, araçá-boi e açaí. Posteriormente, armazenadas em câmara
	fria a 5°C, utilizando embalagem de garrafas pet para maturação, avaliações físico-
	químicas, sensoriais, aspecto global e intenção de compra. As bebidas fermentadas
	foram avaliadas por 80 provadores não treinados do referido Campus através de
	uma escala hedônica de aceitabilidade, cujas notas variaram de 1 a 9, desgostei
	muitíssimo a gostei muitíssimo respectivamente. Em geral, os valores médios para
	as bebidas fermentadas obtiveram notas entre 7 a 8 pontos quanto aos quesitos de
	sabor, aroma, textura, cor e aspecto global. Através da pesquisa, 52\% dos
	provadores não demonstraram conhecimento dos produtos derivados de soro do
	leite. O soro de leite bovino é uma possibilidade de aplicação de proteínas como
	ingredientes funcionais. A elaboração de bebidas lácteas com o aproveitamento da
	matéria prima regional se destacou um grande potencial econômico para geração de
	renda na Amazônia Ocidental. No Brasil, por exemplo, o soro resultante da indústria
	queijeira é utilizado esporadicamente como alimentação para a criação animal.
	
	\vspace{\onelineskip}
	
	\noindent
	\textbf{Palavras-chave}: Consumo. Fermentação. Produto.
	
\end{document}