\documentclass[article,12pt,onesidea,4paper,english,brazil]{abntex2}

\usepackage{lmodern, indentfirst, nomencl, color, graphicx, microtype, lipsum}			
\usepackage[T1]{fontenc}		
\usepackage[utf8]{inputenc}		

\setlrmarginsandblock{2cm}{2cm}{*}
\setulmarginsandblock{2cm}{2cm}{*}
\checkandfixthelayout

\setlength{\parindent}{1.3cm}
\setlength{\parskip}{0.2cm}

\SingleSpacing

\begin{document}
	
	\selectlanguage{brazil}
	
	\frenchspacing 
	
	\begin{center}
		\LARGE SHOW DA ROBÓTICA: APLICAÇÃO DE ATIVIDADES LÚDICAS PARA DESENVOLVIMENTO DE PROTÓTIPOS E MOSTRA DE ROBÔS
		
		\normalsize
		Danielle Menezes Marrieli\footnote{Bolsista (Iniciação Científica Júnior), dani.teceletro2016@gmail.com, \textit{Campus} Porto Velho Calama.} 
		Willians de Paula Pereira\footnote{Orientador, willians.pereira@ifro.edu.br, \textit{Campus} Porto Velho Calama.} 
	\end{center}
	
	\noindent A utilização da robótica como ferramenta de ensino está cada vez mais popular na
	comunidade acadêmica, sendo usada com mais frequência por professores das áreas
	mais voltadas ao desenvolvimento tecnológico nos campos da programação,
	eletrônica e automação, entretanto, não significa que esses são seus únicos campos
	de atuação. Esse projeto tem a intenção criar uma atividade lúdica para aproximar as
	pessoas dos conteúdos de robótica educacional, nesse sentido o objetivo principal do
	projeto é desenvolver protótipos que possam interagir diretamente com a comunidade	
	acadêmica de uma forma lúdica e dinâmica, estimulando o desenvolvimento técnico-
	cientifico dos alunos por meio de temas frequentemente abordados pelo público alvo.	
	O projeto é desenvolvido no Laboratório de Pesquisa do Grupo de Pesquisa
	GPMecatrônica no IFRO – Campus Porto Velho Calama, onde o pesquisador deverá
	realizar o levantamento das principais características que diferenciam um ser humano
	de um zumbi, desenvolvendo, assim, um circuito eletrônico que atue na distinção dos
	mesmos e programar o equipamento para que seja realizada a identificação e
	mostrada as características relevantes de cada um. Também deverá desenvolver um
	circuito eletrônico de iluminação que possa ser controlado via rede wifi e adaptado
	para a comunicação com um mostrador, que irá apresentar via rede, textos escritos
	no dispositivo de controle, que classificam as características distintas entre os seres
	humanos vivos e os seres não vivos, identificadas pelo equipamento e ligadas a
	comunicação wifi para mostrar os dados. Esses protótipos desenvolvidos servirão
	como base para o desenvolvimento de futuros projetos de robótica que possam
	abordar diversos outros temas e áreas de interesse da comunidade, que estimulem
	ainda mais a busca por conhecimento e a integração do público aos diversos campos
	de abrangência da tecnologia com os mesmos objetivos e a espera de maiores
	resultados.
	
	\vspace{\onelineskip}
	
	\noindent
	\textbf{Palavras-chave}: Robótica. Equipamento. Zumbi.
	
\end{document}