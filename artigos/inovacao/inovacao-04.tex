\documentclass[article,12pt,onesidea,4paper,english,brazil]{abntex2}

\usepackage{lmodern, indentfirst, nomencl, color, graphicx, microtype, lipsum}			
\usepackage[T1]{fontenc}		
\usepackage[utf8]{inputenc}		

\setlrmarginsandblock{2cm}{2cm}{*}
\setulmarginsandblock{2cm}{2cm}{*}
\checkandfixthelayout

\setlength{\parindent}{1.3cm}
\setlength{\parskip}{0.2cm}

\SingleSpacing

\begin{document}
	
	\selectlanguage{brazil}
	
	\frenchspacing 
	
	\begin{center}
		\LARGE CARRINHO GARÇOM\footnote{Trabalho realizado dentro da Robótica, Automação e Mecatrônica.}
		
		\normalsize
		Amanda Bassani Mendonça\footnote{Bolsista, amanda.mendonca.am3@gmail.com, Campus Porto Velho Calama.} 
		Daniele Bernardo Batista\footnote{Bolsista, danielebernardo435@gmail.com, Campus Porto Velho Calama.} 
		Elen Camanho Antunes\footnote{Bolsista, camanhoelen@gmail.com, Campus Porto Velho Calama.} 
		Matheus Lemes de Holanda\footnote{Bolsista, ma.le.ho2@hotmail.com, Campus Porto Velho Calama.}
		Ricardo Bussons da Silva\footnote{Orientador, ricardo.bussons@ifro.edu.br, Campus Porto Velho Calama} 
	\end{center}
	
	\noindent Visando a diminuição do fluxo desnecessário de pessoas em frente à mesa diretiva
	em palestras, o projeto consiste na construção de um robô para a realização da
	função de um garçom, sendo utilizado inicialmente apenas em eventos para servir
	água aos palestrantes. Para isso, o robô conta com um chassi feito em acrílico;
	motores e rodas para a sua locomoção nos auditórios; sensores de cor para detectar
	linhas de sinalização, sendo pretas para indicar que ele deve seguir e verdes para
	apontar que ele deve efetuar uma curva; um sensor ultrassônico para captar
	obstáculos e, quando isso ocorrer, permitir que ele pare para a pessoa pegar um
	copo com água, localizado em uma bandeja presente na parte superior do robô; e
	uma bateria sendo a fonte de alimentação para todo o funcionamento do robô. A
	parte principal do carrinho, o circuito eletrônico, é composto por um micro
	controlador PIC, programado em linguagem C; um CI L298N para controlar a
	velocidade e o sentido dos motores usados; um regulador de tensão e resistores. A
	montagem e programação do robô foi feita no laboratório do Grupo de Pesquisa
	Mecatrônica, onde a equipe reuniu-se e foi devidamente orientada. Apesar do
	protótipo ainda não ter sido manuseado em um evento oficial, foi comprovada sua
	eficiência em pistas de teste disponíveis no laboratório utilizado. Como resultado,
	não somente ganha-se o aprendizado, como também a concepção de uma inovação
	tecnológica através da oportunidade de participação em um projeto de pesquisa e
	em um evento expositivo para a demonstração do trabalho já desenvolvido até o
	evento e implementação de melhorias e adaptações para novas funcionalidades.
	
	\vspace{\onelineskip}
	
	\noindent
	\textbf{Palavras-chave}: Robô. Garçom. Evento.
	
\end{document}