\documentclass[article,12pt,onesidea,4paper,english,brazil]{abntex2}

\usepackage{lmodern, indentfirst, nomencl, color, graphicx, microtype, lipsum}			
\usepackage[T1]{fontenc}		
\usepackage[utf8]{inputenc}		

\setlrmarginsandblock{2cm}{2cm}{*}
\setulmarginsandblock{2cm}{2cm}{*}
\checkandfixthelayout

\setlength{\parindent}{1.3cm}
\setlength{\parskip}{0.2cm}

\SingleSpacing

\begin{document}
	
	\selectlanguage{brazil}
	
	\frenchspacing 
	
	\begin{center}
		\LARGE DESENVOLVIMENTO DE UM INSTRUMENTO DE MOVIMENTAÇÃO DE
		AMOSTRAS EMISSORAS DE RAIO GAMA PARA DETECTORES HPGe\footnote{Trabalho realizado dentro da área de Conhecimento CNPq: Ciência da Computação (1030007) com
		financiamento do instituto Federal de Rondônia - IFRO.}
		
		\normalsize
		Josileno Roberto da Silva\footnote{Bolsista (modalidade), josilenoroberto@hotmail.com, \textit{Campus} Vilhena.} 
		Erick Leonardo Weil\footnote{Colaborador (a), erick.weil@ifro.edu.br, \textit{Campus} Vilhena.} 
		Roberto Simplício Guimarães\footnote{Orientador (a), roberto.simplicio@ifro.edu.br, \textit{Campus} Vilhena.} 
		Clayton Ferraz Andrade\footnote{Co-orientador, clayton.andrade@ifro.edu.br, \textit{Campus} Ji-Paraná.} 
	\end{center}
	
	\noindent Um problema comum em espectroscopia gama é o posicionamento preciso da
	amostra, já que variações nesse sentido podem levar a alterações na eficiência
	geométrica do sistema. Isso, aliado à praticidade e ganhos de produtividade que
	podem ser obtidos com um sistema que troque automaticamente as amostras
	conforme sua medição termine, leva ao desenvolvimento de sistemas trocadores de
	amostras automáticos, onde é necessário grande enfoque no perfeito
	posicionamento das amostras em relação ao detector. O instrumento de
	movimentação de amostras é um mecanismo automatizado que seleciona uma
	amostra de radioisótopo retirada de um compartimento com outras amostras,
	posicionando-a perpendicularmente sobre a cápsula do detector semicondutor de
	germânio Hiper-Puro (HPGe), incumbido pela aquisição de dados das emissões
	gamas oriundas da amostra. Durante a aquisição, a amostra gira
	perpendicularmente em relação ao seu eixo, reduzindo assim a influência de sua
	não uniformidade. O sistema será controlado por um circuito eletrônico, responsável
	pelos movimentos descritos, bem como, pelo envio do posicionamento da amostra
	ao programa de aquisição (GEN2000). Um programa computacional será instalado
	no computador que por meio de uma interface gráfica amigável o usuário criará e
	manipulará a sua rotina de aquisições de dados de forma automatizada, fornecendo
	praticidade à sua rotina de trabalho e evitando que o mesmo seja
	desnecessariamente exposto à fonte radioativa ao efetuar a troca de amostras no
	detector. O programa, além de receber os dados do programa de aquisição,
	apresentará em forma de diversos relatórios os dados do GEN2000 com os dados
	do posicionador de amostras, podendo ser utilizado para armazenar os dados e
	suas rotinas com o intuito de padronizar o procedimento, assegurando que todas as
	amostras sejam processadas da mesma maneira, resultando na comparabilidade
	das análises e no aumento da reprodutibilidade, conforme a necessidade do
	pesquisador. O presente trabalho está em fase de construção do mecanismo, sendo
	que sua instalação para aquisição de dados ocorrerá no próximo semestre.
	
	\vspace{\onelineskip}
	
	\noindent
	\textbf{Palavras-chave}: Automação. Radiação. Gama.
	
\end{document}