\documentclass[article,12pt,onesidea,4paper,english,brazil]{abntex2}

\usepackage{lmodern, indentfirst, nomencl, color, graphicx, microtype, lipsum, textcomp}			
\usepackage[T1]{fontenc}		
\usepackage[utf8]{inputenc}		

\setlrmarginsandblock{2cm}{2cm}{*}
\setulmarginsandblock{2cm}{2cm}{*}
\checkandfixthelayout

\setlength{\parindent}{1.3cm}
\setlength{\parskip}{0.2cm}

\SingleSpacing

\begin{document}
	
	\selectlanguage{brazil}
	
	\frenchspacing 
	
	\begin{center}
		\LARGE ADUBAÇÃO DO CAPIM \textit{PANICUM MAXIMUM} CV. MOMBAÇA COM SORO DE LEITE\footnote{Trabalho realizado dentro da área de Conhecimento CNPq: Forragicultura.}
		
		\normalsize
		Angel Brenda Bueno dos Santos\footnote{Bolsista (PIBIC), brendabueno8@gmail.com, \textit{Campus} Colorado do Oeste.} 
		Guilherme Peiter Pires\footnote{Colaborador(a), peiterpires@gmail.com, \textit{Campus} Colorado do Oeste.} 
		Marcos Aurélio Anequine de Macedo\footnote{Orientador(a), marcos.anequine@ifro.edu.br, \textit{Campus} Colorado do Oeste.}
	\end{center}
	
	\noindent A adubação orgânica é capaz de proporcionar aumento na produtividade do capim
	Mombaça, além de melhorar as características químicas do solo. Existem fontes
	alternativas que podem substituir a adubação química em sistemas de pastagens. O
	soro de leite, advindo da fabricação dos queijos, é uma alternativa economicamente
	viável, visto que este é um resíduo que pode ser adquirido facilmente em laticínios
	da região. Deste modo, o objetivo do presente trabalho foi avaliar a biomassa verde
	e matéria seca do capim \textit{Panicum maximum} cv. Mombaça adubado com soro de
	leite. O experimento foi conduzido na Chácara Boa Esperança, zona rural de
	Colorado do Oeste (RO) em área de pastagem de capim Mombaça implantada há
	cerca de 4 anos. Houve o corte de uniformização a 0,45 m de altura e a delimitação
	das parcelas, em delineamento blocos casualisados (DBC), com 8 tratamentos e 3
	repetições. A área experimental foi de 216 m$^2$ com parcelas de 9 m$^2$ (3m x 3m).
	Foram utilizadas três doses de adubação nitrogenada (66, 133 e 200 kg de ureia/ha)
	e quatro doses de soro de leite (66, 133, 200 e 267 m$^3$/ha) divididas em quatro
	aplicações, ocorrendo uma vez por mês durante os meses de maio a agosto. A
	avaliação foi realizada com o corte do capim a 45 cm do solo em 1m$^2$ da região
	central da parcela. A metade da amostra coletada foi pesada e secada a
	temperatura de 65° por 72 horas em estufa de circulação de ar forçado, e a outra
	fez-se a separação morfológica e logo depois passou pelo mesmo procedimento de
	secagem. Os dados obtidos foram analisados utilizando o programa de estatística
	ASSISTAT e as médias foram comparadas por meio da utilização do teste de Skott e
	Knott. A aplicação das diferentes doses de soro de leite e ureia no capim Mombaça
	não diferiram entre si quanto a disponibilidade de matéria seca, biomassa e relação
	folha/material morto, possivelmente por as adubações ocorrerem no período de seca
	da região, e não houver quantidade de cortes suficientes para que o capim
	expressasse produção real.
	
	\vspace{\onelineskip}
	
	\noindent
	\textbf{Palavras-chave}: Adubação. Pastagem. Resíduo de laticínio.
	
\end{document}