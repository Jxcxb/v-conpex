\documentclass[article,12pt,onesidea,4paper,english,brazil]{abntex2}

\usepackage{lmodern, indentfirst, nomencl, color, graphicx, microtype, lipsum}			
\usepackage[T1]{fontenc}		
\usepackage[utf8]{inputenc}		

\setlrmarginsandblock{2cm}{2cm}{*}
\setulmarginsandblock{2cm}{2cm}{*}
\checkandfixthelayout

\setlength{\parindent}{1.3cm}
\setlength{\parskip}{0.2cm}

\SingleSpacing

\begin{document}
	
	\selectlanguage{brazil}
	
	\frenchspacing 
	
	\begin{center}
		\LARGE CARACTERIZAÇÃO FITOQUÍMICA DO EXTRATO ALCOÓLICO DO CIPÓ-TIMBÓ (DERRIS RARIFLORA)
		
		\normalsize
		Keith Hellen Alves Martins\footnote{Bolsista, (Licenciatura em Química), kehellenmartins@gmail.com, Campus Jí-Paraná.} 
		Ana Caroline Cortês Soares\footnote{Bolsista, (Técnico em Química), Campus Jí-Paraná.} 
		Renato André Zan\footnote{Coordenador, Professor EBTT de Química email: renato.zan@ifro.edu.br , Campus Ji-Paraná.} 
	\end{center}
	
	\noindent Introdução: O cipó timbó é uma planta que há muito tempo é utilizada pelos índios
	na pesca, onde eles batem seus cipós nas águas de rios ou lagoas até soltarem
	uma substância leitosa que causa a morte dos peixes, facilitando a sua captura.
	Apesar dessa substância ser nociva aos peixes não se tem relato de contaminação
	de seres humanos que venham a comer esses peixes. Não se sabe ao certo como
	isso ocorre, ou quais são todos os compostos presentes nessa planta. As
	substâncias que se tem mais conhecimento presente nas espécies da família do
	timbó são os rotenóides, que por muito tempo foram utilizadas para o combate de
	pragas em plantações, o uso dessas plantas foi tão intenso que vários países para
	facilitar seu acesso a elas decidiram importa-las e fazer suas próprias plantações.
	Materiais e métodos: De início foi coletado as amostras de folhas e caules do cipó-
	timbó (derris rariflora), em seguida as amostras foram higienizada e pesadas. Com
	isso adicionou álcool etílico para que fosse possível filtrar as amostras e o filtrado foi
	transferido para um frasco âmbar, colocou-se o filtrado no evaporador rotativo para
	retira o álcool contido na amostra com a finalidade de obter o extrato puro. Com o
	extrato puro realizou-se a fase da extração utilizando hexano, diclorometano e
	acetato de etila na proporção 7/3. Feito isso montou-se um placa. Resultados:
	Apesar das pesquisas de produtos naturais terem aumentado nos últimos tempos,
	pouco ainda se sabe sobre a família do timbó, sobre as outras substancias
	presentes na sua constituição. Com a realização dessa pesquisa pode ser elucidada
	algumas das estruturas encontradas no metabolismo secundário dessas plantas,
	através da realização de cromatografia em coluna e a utilização do CGMS. O
	presente projeto teve como finalidade a coleta e análises estrutural do material
	vegetal escolhido, utilizando os métodos propostos por BARBOSA e CHOZE. Com o
	auxílio desses procedimentos foi possível identificar algumas substâncias que estão
	presentes nos metabolismos dessa espécie, como as variações de porfirinas. Com
	isso pode-se observar que alguns compostos apresentaram probabilidade de 100\%,
	porém não e possível afirmar que as moléculas sugeridas pelo programa seja
	realmente elas. Conclusão: Portanto, novas análises serão necessárias, como RMN
	de $^{1}$H e $^{13}$C, Infravermelho entre outras.
	
	\vspace{\onelineskip}
	
	\noindent
	\textbf{Palavras-chave}: Timbó. Extrato vegetal. CGMS.
	
\end{document}