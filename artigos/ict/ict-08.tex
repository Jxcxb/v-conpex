\documentclass[article,12pt,onesidea,4paper,english,brazil]{abntex2}

\usepackage{lmodern, indentfirst, nomencl, color, graphicx, microtype, lipsum}			
\usepackage[T1]{fontenc}		
\usepackage[utf8]{inputenc}		

\setlrmarginsandblock{2cm}{2cm}{*}
\setulmarginsandblock{2cm}{2cm}{*}
\checkandfixthelayout

\setlength{\parindent}{1.3cm}
\setlength{\parskip}{0.2cm}

\SingleSpacing

\begin{document}
	
	\selectlanguage{brazil}
	
	\frenchspacing 
	
	\begin{center}
		\LARGE ATIVIDADE ALELOPÁTICA DE CYMBOPOGON CITRATUS EM SEMENTES DE BIDENS PILOSA L. E ZEA MAYS L.\footnote{Trabalho realizado dentro das Ciências Agrárias com financiamento do CNPq.}
		
		\normalsize
		Lara Vieira Vilela\footnote{Lara Vieira Vilela (PIBIC-AF), vilelalara.agro@gmail.com, Campus – Colorado do Oeste.} 
		Millene Gonçalves Mangueira\footnote{Millene Gonçalves Mangueira, millenevonrondon@gmail.com, Campus – Colorado do Oeste.} 
		Marcos Aurélio Anequine Macedo\footnote{Marcos Aurélio Anequine Macedo, marcos.anequine@ifro.edu.br, Campus – Colorado do Oeste.} 
		Hugo de Almeida Dan\footnote{Hugo de Almeida Dan, halmeidadan@gmail.com, Campus – Colorado do Oeste.} 
	\end{center}
	
	\noindent O manejo inadequado de plantas daninhas tem propiciado a resistência dessas
	invasoras aos herbicidas disponíveis no mercado, tornando o gasto com o seu
	controle o principal fator no custo da produção do milho. Dessa forma, é necessária
	a adoção de medidas mitigadoras alternativas como, por exemplo, o uso da
	alelopatia por meio de plantas medicinais. A partir desses fatos, procurou-se avaliar
	o efeito alelopático do extrato de capim-santo sobre a germinação e
	desenvolvimento de sementes de picão-preto e milho. O experimento foi conduzido
	em laboratório e casa de vegetação. As folhas de capim-santo foram colhidas e
	submetidas à estufa de circulação forçada por 72 horas. Após a retirada da estufa, o
	material foi moído, pesado e colocado em um Becker com água e levado ao fogo
	para a produção do extrato 100\%. A partir desse extrato foi elaborado concentrações
	de 75, 50 e 25\%. O delineamento adotado foi inteiramente casualizado dispondo de
	4 tratamentos, 4 repetições e 1 testemunha para cada espécie avaliada, e para as
	duas etapas do experimento. As sementes de milho foram dispostas em papeis
	germitest, pré-pesados e umedecidos com água destilada. Cada repetição conteve 3
	papeis germitest, com 100 sementes. As sementes de picão-preto foram dispostas
	em placas de petri sobrepostas em 2 papeis germitest, com 100 sementes para cada
	placa. Em casa de vegetação, o experimento foi implantado em vasos de plásticos
	com a adição de adubo químico de acordo com a recomendação técnica do milho.
	Devido à dormência das sementes de picão-preto e a dificuldade em quebrá-la, não
	foi possível constatar de fato o desenvolvimento das sementes. Esse problema pôde
	ser constatado, também, nos ensaios em casa de vegetação. Onde não foi possível
	observar a presença do picão-preto nem mesmo na testemunha, somente outras
	espécies de plantas invasoras. Por outro lado, foi possível observar, nas duas
	etapas do experimento, que em nenhuma concentração o milho apresentou qualquer
	alteração, em virtude do extrato, na sua germinação e desenvolvimento em sua fase
	vegetativa.
	
	\vspace{\onelineskip}
	
	\noindent
	\textbf{Palavras-chave}: Milho. Picão-preto. Extrato.
	
\end{document}