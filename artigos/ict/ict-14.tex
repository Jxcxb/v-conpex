\documentclass[article,12pt,onesidea,4paper,english,brazil]{abntex2}

\usepackage{lmodern, indentfirst, nomencl, color, graphicx, microtype, lipsum}			
\usepackage[T1]{fontenc}		
\usepackage[utf8]{inputenc}		

\setlrmarginsandblock{2cm}{2cm}{*}
\setulmarginsandblock{2cm}{2cm}{*}
\checkandfixthelayout

\setlength{\parindent}{1.3cm}
\setlength{\parskip}{0.2cm}

\SingleSpacing

\begin{document}
	
	\selectlanguage{brazil}
	
	\frenchspacing 
	
	\begin{center}
		\Large CARACTERÍSTICAS MORFOLÓGICAS E PRODUTIVAS DO CAPIM-PIATÃ EM
		SISTEMA DE ILPF SOB CONDIÇÕES DE SOMBREAMENTO EM DISTINTAS
		ORIENTAÇÕES DE PLANTIO DOS RENQUES DE EUCALIPTO
		
		\normalsize
		Amanda Bonifacio Maciel\footnote{Discente do Curso de Engenharia Agronômica. Bolsista CNPq, Modalidade PIBITI – IFRO \textit{Campus} Colorado do Oeste. E-mail: amanda.maciel295@gmail.com} 
		Paulo Ricardo de Oliveira\footnote{Discente do Curso Técnico em Agropecuária. Bolsista CNPq e IFRO, Modalidade PIBIC - EM – \textit{Campus} Colorado do Oeste.} 
		Ana Paula Silva Melo\footnote{Discente do Curso de Engenharia Agronômica. IFRO \textit{Campus} Colorado do Oeste. E-mail: anap.silvamelo@gmail.com} 
		Rafael Henrique Pereira dos Reis\footnote{Professor do IFRO \textit{Campus} Colorado do Oeste. Docente Orientador. E-mail: rafael.reis@ifro.edu.br}
		Ernando Balbinot\footnote{Professor do IFRO Campus Colorado do Oeste. Docente Co-orientador.} 
	\end{center}
	
	\noindent O sistema integração Lavoura-Pecuária-Floresta (iLPF) pode ser uma alternativa na
	recuperação de áreas degradadas. Entretanto, a forrageira utilizada pode ter sua
	produção influenciada pela interação dos componentes, principalmente as árvores,
	sobretudo em condições de sombreamento excessivo. A disposição de renques do
	componente arbóreo no espaço geográfico pode, nesse sentido, proporcionar maior
	ou menor sombreamento na área e, por consequência, sobre o sistema integrado.
	Objetivou-se com esse trabalho investigar o comportamento da forrageira \textit{Brachiaria
	brizantha} cv. BRS Piatã sob sombreamento de eucalipto plantado em diferentes
	orientações de renques. O experimento foi conduzido em delineamento de blocos
	casualizados em esquema de parcelas subdivididas (3x3), sendo a parcela a
	disposição/orientação dos renques de eucalipto (sentidos Leste-Oeste, Norte-Sul e
	declive do terreno delimitado pelos terraços), e a subparcela pelas faixas paralelas
	de distância em relação ao renque de eucalipto (0 a 8,6 m; 8,7 a 17,3 m e 17,4 a 26
	m). Realizaram-se avaliações da altura do dossel forrageiro ao final de cada ciclo
	(quando atingiu altura de corte, simulando a entrada dos animais para pastejo),
	número de perfilhos m$^{-2}$, e ainda, foram coletadas amostras utilizando quadro de
	amostragem (1,0 m$^{2}$) para determinação da produtividade por hectare e
	subamostras para determinação da porcentagem de matéria seca (MS) e proporção
	de componentes (lâmina foliar, pseudocolmo e material morto). Os dados foram
	submetidos à análise de variância e, quando houve efeito significativo, foi aplicado o
	teste de comparação de médias de Tukey a 5\% de probabilidade de erro. Como
	estava no primeiro ano de implantação do componente arbóreo, o sombreamento
	promoveu pequenas alterações sobre a forrageira. A exemplo, os teores de MS e
	participação de folhas mortas, onde as médias não diferiram entre si
	estatisticamente. Ainda assim, observou-se diferenças nas variáveis de altura
	(média de 71 cm), produtividade (média de 11 t ha$^{-1}$), número de perfilhos e participação do colmo (35\%), onde foram superiores em condições sombreadas, sobretudo na orientação Norte-Sul. Entretanto, a participação de folhas verdes foi
	superior na orientação Leste-Oeste (86\%). Portanto, pode-se inferir que essa
	forrageira, na sombra, tende a alongar seu colmo, dado a busca por luminosidade,
	fator indispensável aos processos fotossintéticos.
	
	\vspace{\onelineskip}
	
	\noindent
	\textbf{Palavras-chave}: Agrossilvipastoril. Disposição geográfica. Renque.
	
\end{document}