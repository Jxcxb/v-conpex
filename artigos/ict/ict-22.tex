\documentclass[article,12pt,onesidea,4paper,english,brazil]{abntex2}

\usepackage{lmodern, indentfirst, nomencl, color, graphicx, microtype, lipsum, textcomp}			
\usepackage[T1]{fontenc}		
\usepackage[utf8]{inputenc}		

\setlrmarginsandblock{2cm}{2cm}{*}
\setulmarginsandblock{2cm}{2cm}{*}
\checkandfixthelayout

\setlength{\parindent}{1.3cm}
\setlength{\parskip}{0.2cm}

\SingleSpacing

\begin{document}
	
	\selectlanguage{brazil}
	
	\frenchspacing 
	
	\begin{center}
		\LARGE ENTRE O REGIONAL E O GLOBAL: UM ESTUDO DOS DESAFIOS
		PERTINENTES À EMERGÊNCIA DO ZIKA VÍRUS NA AMAZÔNIA BRASILEIRA\footnote{Trabalho realizado dentro da Ciências Humanas, com financiamento do CNPq e PROPESP/IFRO.}
		
		\normalsize
		Larah Drielly Santos Herrera\footnote{Bolsista PIBITI/IFRO, herrera.larah@gmail.com, \textit{Campus} Colorado do Oeste.} 
		Diego Ferreira dos Santos\footnote{Colaborador PIBITI/IFRO, diego.fds95@gmail.com, \textit{Campus} Colorado do Oeste.} 
		Dany Roberta Marques Caldeira\footnote{Orientadora, dany.caldeira@ifro.edu.br, \textit{Campus} Colorado do Oeste.} 
		Ernando Balbinot\footnote{Co-orientador, ernando.balbinot@ifro.edu.br, \textit{Campus} Colorado do Oeste.} 
	\end{center}
	
	\noindent A diferença no comportamento nutricional entre os híbridos de eucalipto, apresenta
	uma grande relevância prática, pois permite a alocação desses materiais em solos
	de baixa fertilidade natural e a adoção de regimes diferenciados de adubação.
	Diante do exposto, objetivou-se avaliar o estado nutricional de seis híbridos de
	Eucalyptus sp., aos 18 meses de idade, no cone sul do estado de Rondônia. O
	presente trabalho foi conduzido na Unidade de Referencia Tecnológica do Instituto
	Federal de Educação, Ciência e Tecnologia de Rondônia, \textit{Campus} Colorado do
	Oeste. Os tratamentos consistiram em seis clones de  \textit{Eucalyptus} cultivados em
	sentido direcionado Leste/Oeste e Norte/Sul em sistema de iLPF. O espaçamento
	adotado foi de três metros entre linhas no renque, com dois metros de distâncias
	entre arvores e vinte e seis metros entre renques. O experimento foi conduzido em
	delineamento experimental de blocos casualizados com cinco blocos e 10 repetições
	em cada bloco. Em cada parcela de 25 árvores, para fins da avaliação nutricional
	foliar, foram amostrados seis indivíduos, retirando-se os galhos do terço médio onde
	foram coletadas as folhas 3, 4, 5 e 6 a partir do ápice e recém-maduras, totalizando
	uma amostra composta com 100 folhas por cada parcela experimental. As folhas
	foram armazenadas em sacos de papel e conduzidas para laboratório e,
	posteriormente foram lavadas com água destilada, colocadas em estufa de
	circulação forçada a 75 °C e moídas com peneira de 1 mm em moinho do tipo
	Willye. Os resultados obtidos foram submetidos a analise de variância e, para os
	efeitos significativos de tratamento, foram aplicadas as médias ao teste de Tukey, a
	1\% de probabilidade de erro. Considerando as médias das concentrações foliares de
	nutrientes, os clones de \textit{Eucalyptus} estudados, apresentam a mesma ordem de
	assimilação para os macronutrientes: N=P=K=Ca=Mg=S. Para os micronutrientes,
	todos apresentam a mesma assimilação, exceto o Zn: B=Cu=Fe=Mn>Zn. As
	concentrações foliares dos nutrientes relacionadas ao desenvolvimento das plantas
	indicam entre outras coisas a capacidade de assimilação e eficiência na utilização
	de nutrientes. Este trabalho foi feito após 18 meses de plantio e mostra a
	necessidade em se dar continuidade às avaliações para resultados mais concisos e
	precisos.
	
	\vspace{\onelineskip}
	
	\noindent
	\textbf{Palavras-chave}: Híbridos. Seleção. Genética.
	
\end{document}