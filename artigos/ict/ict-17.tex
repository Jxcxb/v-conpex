\documentclass[article,12pt,onesidea,4paper,english,brazil]{abntex2}

\usepackage{lmodern, indentfirst, nomencl, color, graphicx, microtype, lipsum}			
\usepackage[T1]{fontenc}		
\usepackage[utf8]{inputenc}		

\setlrmarginsandblock{2cm}{2cm}{*}
\setulmarginsandblock{2cm}{2cm}{*}
\checkandfixthelayout

\setlength{\parindent}{1.3cm}
\setlength{\parskip}{0.2cm}

\SingleSpacing

\begin{document}
	
	\selectlanguage{brazil}
	
	\frenchspacing 
	
	\begin{center}
		\LARGE DESEMPENHO DO FEIJOEIRO À INOCULAÇÃO E ADUBAÇÃO NITROGENADA
		E MOLÍBDICA EM CONDIÇÕES AMAZÔNICAS\footnote{Trabalho realizado dentro da (área de Conhecimento CNPq: Ciências Agrárias) com financiamento do (CNPq e IFRO).}
		
		\normalsize
		Ivanildo Guilherme Henrique\footnote{Bolsista (PIBIC-Af), ivanildo.guilhermee@gmail.com, \textit{Campus} Colorado do Oeste.} 
		Natielly Silva Cardoso\footnote{Bolsista (PIBIC-EM), natielly.ifro@gmail.com, \textit{Campus} Colorado do Oeste.} 
		Marcos Aurélio Anequine Macedo\footnote{Orientador, marcos.anequine@ifro.edu.br, \textit{Campus} Colorado do Oeste.} 
		Luiz Cobiniano de Melo Filho\footnote{Co-orientador, luiz.cobiniano@ifro.edu.br, \textit{Campus} Colorado do Oeste.} 
	\end{center}
	
	\noindent O feijão comum (\textit{Phaseolus vulgaris} L.) é um dos principais componentes da dieta alimentar brasileira, constituindo uma das mais importantes fontes de proteína
	vegetal, sobretudo para a população de baixa renda. Rondônia é o estado da região
	norte com maior área cultivada com feijão, todavia, o rendimento encontra-se entre
	os menores, quando comparado com o dos principais estados produtores do país. O
	feijoeiro é uma cultura exigente em nutrientes, sendo o nitrogênio (N) o mais
	requisitado. Em função das perdas, poluição ambiental e elevado custo dos
	fertilizantes nitrogenados, a fixação biológica de N (FBN) se apresenta com possível
	solução. O molibdênio (Mo), presente nas enzimas nitrogenase e redutase do
	nitrato, é muito importante no metabolismo do nitrogênio, inclusive da FBN. Desse
	modo, o trabalho foi conduzido na área experimental do Instituto Federal de
	Educação Ciência e Tecnologia de Rondônia, \textit{Campus} Colorado do Oeste, com
	objetivo de avaliar o efeito da inoculação com \textit{Azospirillum brasilense} e \textit{Rhizobium tropici}, isolados e em conjunto e as possíveis interações com a adubação
	nitrogenada e molíbdica em cobertura, no desenvolvimento e produtividade do
	feijoeiro. Sendo delineamento experimental utilizado o de blocos casualizados
	disposto em esquema fatorial 4x2x2. Os tratamentos constituídos pela combinação
	da inoculação de sementes; aplicação foliar de molibdênio; e fornecimento de
	nitrogênio em cobertura. Os dados foram submetidos à análise de variância, sendo
	as médias dos diferentes tratamentos comparadas pelo teste de Tukey. Após
	aquisição e análise dos dados evidenciou-se que os tratamentos não influenciaram
	no que se refere aos teores de clorofila, assim como não expressou alterações
	significativas no estande final de plantas. Para as variável Número de vagens por
	planta houve significância. É possível observar, a diferença entre as médias dos
	tratamentos com e sem aplicação de nitrogênio em cobertura no que se refere ao
	número de vagens. Sendo os valores superiores constatados com a aplicação de
	fertilizante nitrogenado. Isso deve-se a uma possível maior área foliar e
	fotossintética proporcionada pelo fertilizante.
	
	\vspace{\onelineskip}
	
	\noindent
	\textbf{Palavras-chave}: \textit{Azospirillum brasilense}. Fixação biológica. Molibdênio. \textit{Rhizobium tropici}
	
\end{document}