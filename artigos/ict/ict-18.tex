\documentclass[article,12pt,onesidea,4paper,english,brazil]{abntex2}

\usepackage{lmodern, indentfirst, nomencl, color, graphicx, microtype, lipsum}			
\usepackage[T1]{fontenc}		
\usepackage[utf8]{inputenc}		

\setlrmarginsandblock{2cm}{2cm}{*}
\setulmarginsandblock{2cm}{2cm}{*}
\checkandfixthelayout

\setlength{\parindent}{1.3cm}
\setlength{\parskip}{0.2cm}

\SingleSpacing

\begin{document}
	
	\selectlanguage{brazil}
	
	\frenchspacing 
	
	\begin{center}
		\LARGE DESENVOLVIMENTO INICIAL DE \textit{EUCALYPTUS SP.} NO CONE SUL DE RONDÔNIA\footnote{Trabalho realizado dentro da área de Conhecimento CNPq: 5.01.00.00-0 Silvicultura / 5.02.01.05-0	Nutrição florestal, com financiamento do IFRO.}
		
		\normalsize
		Wanderson Junior Lima do Nascimento\footnote{Bolsista PIBITI, wandersonj2018@gmail.com, \textit{Campus} Colorado do Oeste.} 
		Larah Drielly Santos Herrera\footnote{Colaboradora, herrera.larah@gmail.com, \textit{Campus} Colorado do Oeste.} 
		Dany Roberta Marques Caldeira\footnote{Orientadora,dany.caldeira@ifro.edu.br, \textit{Campus} Colorado do Oeste.} 
		Ernando Balbinot\footnote{Co-orientador, ernando.balbinot@ifro.edu.br, \textit{Campus} Colorado do Oeste.} 
	\end{center}
	
	\noindent Plantações florestais proporcionam benefícios econômicos, sociais e ambientais.
	Indivíduos de uma mesma espécie apresentam desenvolvimentos distintos quando
	avaliados em regiões ecológicas diferentes, assim como espécies diferentes de
	eucaliptos podem reproduzir respostas bem variadas em um mesmo sítio florestal.
	Outro fator que deve ser levado em consideração é o direcionamento do plantio,
	uma vez que é de costume adotar o plantio orientado no sentido leste-oeste, devido
	à maior interceptação de radiação solar ao longo dia, porém questões como a
	declividade do terreno devem ser levadas em consideração a fim da utilização de
	boas práticas de manejo e conservação do solo. O objetivo deste trabalho foi avaliar
	o desenvolvimento inicial de clones de \textit{Eucalyptus sp.} no Cone Sul do estado de
	Rondônia em sistema integrado de produção. Os tratamentos consistiram em seis
	clones de Eucalyptus sp. cultivados em dois direcionamentos de plantio, leste-oeste
	e norte-sul. O delineamento experimental foi de blocos casualizados com 5
	repetições. Foram realizadas medições do diâmetro a altura do peito (DAP) e altura
	total da planta (HT), aos 6, 12 e 18 meses de idade. Os resultados obtidos foram
	submetidos a ANOVA e, para os efeitos significativos de tratamento, foram aplicadas
	as médias ao teste de Tukey, a 1\% de probabilidade de erro. Aos seis meses após o
	plantio os clones 2, 4 e 6 obtiveram os melhores resultados para o parâmetro altura,
	para o diâmetro a altura do peito destacaram-se os clones 2 e 4, não houve
	diferença significativa para os direcionamentos dos plantios. Aos 12 meses, não
	houve diferença significativa no crescimento em altura e direcionamento do plantio
	para esta variável, entretanto quando comparados os DAPs médios, os clones 1 e 4
	destacaram-se dos demais, assim como o direcionamento do plantio norte-sul
	apresentou maiores resultados para a mesma variável. Aos 18 meses após o
	plantio, os clones não diferiram significativamente para as variáveis altura e DAP
	isoladamente, entretanto o direcionamento norte-sul influenciou positivamente no
	incremento de altura e DAP. Pode-se notar mudanças responsivas para os
	parâmetros avaliados ao longo do trabalho, o que demonstra a necessidade da sua
	continuidade para a obtenção de dados mais conclusivos.
	
	\vspace{\onelineskip}
	
	\noindent
	\textbf{Palavras-chave}: Plantio direcionado. Crescimento. Respostas.
	
\end{document}