\documentclass[article,12pt,onesidea,4paper,english,brazil]{abntex2}

\usepackage{lmodern, indentfirst, nomencl, color, graphicx, microtype, lipsum, textcomp}			
\usepackage[T1]{fontenc}		
\usepackage[utf8]{inputenc}		

\setlrmarginsandblock{2cm}{2cm}{*}
\setulmarginsandblock{2cm}{2cm}{*}
\checkandfixthelayout

\setlength{\parindent}{1.3cm}
\setlength{\parskip}{0.2cm}

\SingleSpacing

\begin{document}
	
	\selectlanguage{brazil}
	
	\frenchspacing 
	
	\begin{center}
		\LARGE AVALIAÇÕES AGRONÔMICAS DE CULTIVARES DE ALHO SUBMETIDAS A
		VERNALIZAÇÃO EM DIFERENTES ÉPOCAS DO ANO NO CONE SUL DE	
		RONDÔNIA
		
		\normalsize
		Vitor Vicente Klein\footnote{Bolsista (PIBiC - Af), email, vitorvklein@gmail.com, \textit{Campus} Colorado do Oeste Rondônia.} 
		Ezequiel Soares Da Silva\footnote{Colaborador(a), email, ezequielifro@gmail.com, \textit{Campus} Colorado do Oeste Rondônia.} 
		Marcos Aurélio Anequine Macedo\footnote{Orientador(a), email, marcos.anequine@ifro.edu.br, \textit{Campus} Colorado do Oeste Rondônia.} 
		Valdique Gilberto de Lima\footnote{Co-orientador(a), email, valdique.lima@ifro.edu.br \textit{Campus} Colorado do Oeste Rondônia.} 
	\end{center}
	
	\noindent A cultura do alho possui grande importância socioeconômica para o Brasil, devido a
	este estar presente na mesa da maioria dos brasileiros e possuir alta capacidade de
	geração de emprego para a realização de sua produção. Para o estado de
	Rondônia, juntamente com os demais estados da região norte, a produção desta
	cultura é inexistente (proporcionando elevados preços aos consumidores da região),
	devido ao fato da mesma necessitar clima mais ameno para ser produzida
	convencionalmente. Para que a cultura seja produzida em condições de clima
	tropical existe a técnica de vernalização, que corresponde em submeter os bulbilhos
	(dentes) por dias, a temperaturas em torno de 5°C, antes do plantio. Além disso,
	existe no mercado a cultivar BRS Hozan (semi-nobre), ao qual possui capacidade de
	ser produzida sob temperaturas elevadas, proporcionando menor custo de
	instalação, por não necessitar de câmaras frias para a vernalização. Sendo assim, o
	objetivo do trabalho foi avaliar as cultivares de alho semeadas durante os meses de
	março a agosto em diferentes períodos de vernalização. A pesquisa foi realizada a
	campo no setor de Olericultura do Campus Colorado do Oeste, com 7 tratamentos,
	que consistiam em T1: BRS Hozan; T2: Ito sem vernalização; T3: Ito com 20 dias de
	vernalização; T4: Ito com 40 dias de vernalização; T5: Caçador sem vernalização;
	T6: Caçador com 20 dias de vernalização; T7: Caçador com 40 dias de vernalização.
	O delineamento utilizado foi o de blocos casualizados, com quatro repetições. Para o
	mês de março os resultados mais satisfatórios corresponderam ao tratamento 6, ao
	qual mostra superioridade da cultivar caçador com 20 dias de vernalização, com
	produtividade por hectare em torno de 3600 kg. Além disso, os trartamentos
	testemunha (sem vernalização) não apresentaram diferenciação do bulbo (produção
	de bulbilhos), mostrando que há a necessidade da vernalização e que tal técnica
	tem funcionabilidade à região. O resultado obtido mostra que, embora inferior à
	média nacional e aos resultados obtidos em pesquisas realizadas em outras regiões,
	o estado de Rondônia possui capacidade de produzir a cultura do alho.
	
	\vspace{\onelineskip}
	
	\noindent
	\textbf{Palavras-chave}: Alho. Vernalização. Épocas do ano.
	
\end{document}