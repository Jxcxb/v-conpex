\documentclass[article,12pt,onesidea,4paper,english,brazil]{abntex2}

\usepackage{lmodern, indentfirst, nomencl, color, graphicx, microtype, lipsum}			
\usepackage[T1]{fontenc}		
\usepackage[utf8]{inputenc}		

\setlrmarginsandblock{2cm}{2cm}{*}
\setulmarginsandblock{2cm}{2cm}{*}
\checkandfixthelayout

\setlength{\parindent}{1.3cm}
\setlength{\parskip}{0.2cm}

\SingleSpacing

\begin{document}
	
	\selectlanguage{brazil}
	
	\frenchspacing 
	
	\begin{center}
		\LARGE REDUÇÃO NO ESPAÇAMENTO ENTRELINHAS NO CULTIVO DE MILHO EM
		CONSÓRCIO COM CAPIM-MARANDU PARA PRODUÇÃO DE SILAGEM\footnote{Trabalho realizado dentro da área de conhecimento CNPq: Ciências Agrárias.}
		
		\normalsize
		Diego Ferreira dos Santos\footnote{Bolsista PIBITI, diego.fds95@gmail.com, \textit{Campus} Colorado do Oeste.} 
		Gustavo Wender Xavier\footnote{Bolsista ICJ, gusttavoifro@gmail.com, \textit{Campus} Colorado do Oeste.} 
		Natanael Maikon dos Santos\footnote{Bolsista PIBITI, natanaelmaikon42@gmail.com, \textit{Campus} Colorado do Oeste.} 
		Rafael Henrique Pereira dos Reis\footnote{Orientador, rafael.reis@ifro.edu.br, \textit{Campus} Colorado do Oeste.} 
	\end{center}
	
	\noindent O sistema de integração lavoura-pecuária é uma importante ferramenta para
	recuperação de áreas degradadas, cobertura do solo e formação e recuperação de
	pastagens. Um dos consórcios mais utilizados é o de milho com \textit{Brachiaria}, onde o
	espaçamento entre fileiras do milho pode influenciar diretamente no
	desenvolvimento das plantas. A redução desse espaçamento por conta da
	padronização de ajustes nos maquinários e tratos culturais é uma realidade e foi
	nesse sentido que a pesquisa objetivou avaliar o efeito de diferentes espaçamentos
	na semeadura do milho em consórcio com capim \textit{Brachiaria brizantha} cv. Marandu
	sobre a produtividade das culturas e qualidade da silagem. O projeto foi
	desenvolvido no Instituto Federal de Educação, Ciência e Tecnologia de Rondônia -
	Campus Colorado do Oeste, no período compreendido entre agosto de 2016 e julho
	de 2017. O experimento foi conduzido em delineamento em blocos casualizados
	arranjado em fatorial duplo (2x3+5) com quatro repetições, sendo o primeiro fator os
	espaçamentos entre linhas (0,45m e 0,90m), o segundo fator as modalidades de
	semeadura (milho com capim a lanço; milho com capim na linha e milho com capim
	na entrelinha com espaçamento de 0,90m), e as testemunhas com cultivos solteiros
	(milho a 0,45m; milho a 0,90m; capim a 0,45m; capim a 0,90m; capim a lanço).
	Foram avaliadas as variáveis do capim após rebrota: altura de plantas; produtividade
	de massa verde; porcentagem de matéria seca; e proporções do colmo, material
	morto e folhas. Apenas a variável altura de plantas demonstrou diferença
	significativa entre os tratamentos, onde os maiores valores foram encontrados nos
	plantios de milho a 0,45 e 0,90 metro com capim na linha do milho, apresentando
	médias de altura de 0,83 e 0,79 metro, respectivamente. Esse resultado pode estar
	relacionado à disposição das plantas de capim na mesma linha de semeadura do
	milho, o que fez com que as plantas crescessem mais em busca de luz. As demais
	variáveis não foram influenciadas pelos tratamentos, demonstrando que o cultivo do
	capim em consórcio com o milho não interferiu no seu estabelecimento como
	forrageira na formação de pastagens.
	
	\vspace{\onelineskip}
	
	\noindent
	\textbf{Palavras-chave}: ILP. Espaçamento. Marandu.	
	
\end{document}