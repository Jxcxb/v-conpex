\documentclass[article,12pt,onesidea,4paper,english,brazil]{abntex2}

\usepackage{lmodern, indentfirst, nomencl, color, graphicx, microtype, lipsum}			
\usepackage[T1]{fontenc}		
\usepackage[utf8]{inputenc}		

\setlrmarginsandblock{2cm}{2cm}{*}
\setulmarginsandblock{2cm}{2cm}{*}
\checkandfixthelayout

\setlength{\parindent}{1.3cm}
\setlength{\parskip}{0.2cm}

\SingleSpacing

\begin{document}
	
	\selectlanguage{brazil}
	
	\frenchspacing 
	
	\begin{center}
		\LARGE PLANEJAMENTO, SÍNTESE E AVALIAÇÃO FARMACOLÓGICA DE DERIVADOS 1,3,4-OXADIAZOLA-2(3H)-TIONA\footnote{Tese realizada dentro da área da Química Orgânica Medicinal com financiamento da Capes e CNPq.}
		
		\normalsize
		Amanda Feitosa Cidade\footnote{Autora da tese, amanda.cidade@ifro.edu.br, Campus Calama.} 
		Ricardo Menegatti\footnote{Orientador, Instituto de Química, Universidade Federal de Goiás.} 
		Luciano Morais Lião\footnote{Co-Orientador, Faculdade de Farmácia, Universidade Federal de Goiás.}
	\end{center}
	
	\noindent O processo inflamatório é um mecanismo de defesa do organismo frente a estímulos
	nocivos, porém, se persistente contribui para a patogênese de muitas doenças. No
	âmbito de uma linha de pesquisa que visa o desenvolvimento de novos candidatos a
	protótipos de fármacos anti-inflamatórios multialvos, neste estudo é descrito o
	planejamento e a síntese dos compostos 1,3,4-oxadiazola-2(3H)-tiona, bem como a
	avaliação do efeito anti-inflamatório. Estes compostos foram planejados via
	estratégia de hibridação molecular, a partir do protótipo LQFM21, que apresenta
	efeito antinociceptivo e anti-inflamatório em modelos agudos e crônicos, e do
	derivado do ácido flufenâmico, protótipo anti-inflamatório inibidor das vias COX-2 e
	5-LOX. A rota sintética utilizada na obtenção dos compostos alvo, se mostrou
	eficiente e rendimentos globais que variaram entre 22-51\% e atende á princípios da
	química verde. O tratamento dos camundongos reduziu o número de contorções
	abdominais induzidas por ácido acético, sendo que a ação antinociceptiva foi
	confirmada na segunda fase do teste de dor induzida pela formalina, o que permitiu
	dissociar a atividade anti-inflamatória deste composto de uma atividade analgésica
	central. Foi observado redução no edema, no modelo do edema de pata, e a ação
	anti-inflamatória foi confirmada pela redução da migração celular de leucócitos e
	células polimorfonucleares em 43,8\% e 61,8\%, respectivamente, similar ao controle
	positivo dexametasona (60,6\% e 74,7\% de redução). Foi observada a redução da
	atividade da enzima mieloperoxidase (27,8\% de redução), no modelo de pleurisia
	em camundongos, e o tratamento com um dos compostos da série, reduzir a
	concentração de TNF-$\alpha$ no exsudato pleural dos camundongos em 56 \%, o controle
	positivo dexametasona reduziu em 77,5\% comparado ao grupo controle. Em
	conclusão sugere-se que a ação anti-inflamatória do composto aqui estudado, pode
	estar relacionada com a redução dos níveis TNF-$\alpha$. Uma vez que a citocina TNF-$\alpha$ é
	um importante fator na progressão da artrite reumatoide, podemos inferir que o
	composto testado apresenta perfil promissor e pode corroborar no âmbito do
	desenvolvimento de novos protótipos de fármacos relacionados a doenças
	inflamatórias crônicas. A avaliação farmacológica de todos os compostos
	sintetizados neste trabalho servirá de guia no intuito de otimizar a atividade anti-
	inflamatória.
	
	\vspace{\onelineskip}
	
	\noindent
	\textbf{Palavras-chave}: Fármacos. Anti-inflamatórios. Desenvolvimento.
	
\end{document}