\documentclass[article,12pt,onesidea,4paper,english,brazil]{abntex2}

\usepackage{lmodern, indentfirst, nomencl, color, graphicx, microtype, lipsum}			
\usepackage[T1]{fontenc}		
\usepackage[utf8]{inputenc}		

\setlrmarginsandblock{2cm}{2cm}{*}
\setulmarginsandblock{2cm}{2cm}{*}
\checkandfixthelayout

\setlength{\parindent}{1.3cm}
\setlength{\parskip}{0.2cm}

\SingleSpacing

\begin{document}
	
	\selectlanguage{brazil}
	
	\frenchspacing 
	
	\begin{center}
		\LARGE MEMÓRIA COLETIVA, CULTURA E IDENTIDADE: UMA RECONSTRUÇÃO DA
		HISTÓRIA DA COMUNIDADE QUILOMBOLA DE SANTA FÉ DO GUAPORÉ ATRAVÉS DA ORALIDADE\footnote{Trabalho realizado dentro da área “Antropologia das Populações Afro-Brasileiras” com financiamento da Coordenação de Aperfeiçoamento de Pessoal de Nível Superior – CAPES.}
		
		\normalsize
		Cássio Alves Lus\footnote{Mestrando em História e Estudos Culturais pela Fundação Universidade Federal de Rondônia. Professor EBTT de Sociologia do Instituto Federal de Rondônia. Bolsista CAPES. E-mail: cassio.lus@ifro.edu.br} 
		Sérgio Luiz de Souza\footnote{Professor do Departamento de Ciências Sociais da Fundação Universidade Federal de Rondônia. Docente do Programa de Pós-Graduação Stricto Sensu em História e Estudos Culturais. Orientador. E-mail: sergiosouza@unir.br}  
	\end{center}
	
	\noindent Esta pesquisa está contextualizada nas discussões acerca da memória coletiva,
	cultura e da identidade, assim como dos estudos étnico-raciais. Temos como foco a
	reconstrução da história da comunidade quilombola de Santa Fé do Guaporé, no
	Estado de Rondônia, entre os anos de 1988, ano da promulgação da Constituição
	da República Federativa do Brasil, que definiu e reconheceu legalmente o termo
	“remanescentes de comunidades de quilombos”, no artigo 68 do Ato das
	Disposições Transitórias, definindo – os como coletivos humanos que tem direito ao
	reconhecimento e titulação das suas terras, até o ano de 2017, ano do
	reconhecimento por parte do Estado brasileiro ao reconhecimento e titulação do
	território quilombola de Santa Fé do Guaporé. A história oral e a etnografia estão
	sendo utilizadas como recursos teórico-metodológicos fundamentais para
	reconstruirmos a história e interpretarmos as práticas socioculturais da comunidade
	quilombola local. Para tanto, utilizamo-nos de fontes escritas como jornais e fontes
	iconográficas, mas, principalmente, de relatos orais de quilombolas que realizaram
	e/ou vivenciaram o processo histórico dos últimos 30 anos da comunidade. Os
	relatos orais podem ser fontes que trazem aos pesquisadores uma maior
	sensibilidade quanto às múltiplas determinações, à contextualização e às mudanças
	no decorrer do tempo. Assim, nossas lembranças permanecem coletivas, e as
	pessoas nos fazem lembrá-las, mesmo que se trate de acontecimentos nos quais só
	nós estivemos envolvidos e com objetos que só nós vimos. Nessa perspectiva, a
	pesquisa busca reconstruir a história dessa comunidade através dos relatos orais de
	seus moradores, priorizando a produção de conhecimentos sobre esse grupo étnico
	à partir das memórias dos velhos. Com este estudo buscar-se-á subverter a
	narrativa monolítica da história oficial, que apresenta apenas uma versão dos fatos
	sem dar voz aos grupos socialmente e historicamente marginalizados. As
	bibliografias sobre populações negras, dinâmica cultural e nação no Brasil, as
	reflexões antropológicas acerca da relações entre culturas, democracia, e o
	autoritarismo no contexto brasileiro, os estudos sobre memória coletiva, e as
	contribuições e interpretações das obras que abordam os grupos étnicos e suas
	fronteiras, estão sendo utilizadas como referencial teórico-conceitual nessa pesquisa
	para pensarmos o processo social e histórico de produção das diferenças culturais
	das comunidades quilombolas do Vale do Guaporé.
	
	\vspace{\onelineskip}
	
	\noindent
	\textbf{Palavras-chave}: Quilombolas. Memória coletiva. Cultura.
	
\end{document}