\documentclass[article,12pt,onesidea,4paper,english,brazil]{abntex2}

\usepackage{lmodern, indentfirst, nomencl, color, graphicx, microtype, lipsum}			
\usepackage[T1]{fontenc}		
\usepackage[utf8]{inputenc}		

\setlrmarginsandblock{2cm}{2cm}{*}
\setulmarginsandblock{2cm}{2cm}{*}
\checkandfixthelayout

\setlength{\parindent}{1.3cm}
\setlength{\parskip}{0.2cm}

\SingleSpacing

\begin{document}
	
	\selectlanguage{brazil}
	
	\frenchspacing 
	
	\begin{center}
		\LARGE PRÁTICAS PEDAGÓGICAS E FORMAÇÃO DOCENTE: A INCLUSÃO NO IFRO, \textit{CAMPUS} PORTO VELHO CALAMA
		
		\normalsize
		Suelene da Silva Batista\footnote{Mestranda, suelene.batista@ifro.edu.br, IFRO, \textit{Campus} Porto Velho Calama.} 
		Dra. Carmen Tereza Velanga\footnote{Orientadora, carmen.velanga@ifro.edu, UNIR, \textit{Campus} Porto Velho.} 
	\end{center}
	
	\noindent A presente pesquisa foi realizada com objetivo de analisar o processo de inclusão a
	partir das práticas pedagógicas dos professores para o atendimento dos estudantes
	público-alvo da educação especial no IFRO, Campus Porto Velho – Calama, na
	perspectiva de promover ações por meio da reflexão entre os envolvidos no contexto
	da pesquisa, portanto, foram utilizados os pressupostos da pesquisa qualitativa com
	ênfase na pesquisa-ação. Os instrumentos utilizados para a realização da pesquisa
	foram: Análise Documental, visando conhecer a concepção acerca da Inclusão
	estabelecida nos documentos Institucionais; Observação Direta dos espaços
	escolares para obter informações sobre o ambiente pedagógico; Entrevistas
	semiestruturadas dirigidas aos professores e alunos para obter informações sobre
	as ações desenvolvidas e projetadas pela Instituição para o atendimento aos alunos;
	Grupo Focal, com a participação dos servidores do Núcleo de Atendimento às
	Pessoas com Necessidades Educacionais Específicas (NAPNE) e professores, para
	o aprofundamento da situação investigada e proposições de intervenção. Os
	resultados parciais indicaram a formação docente, o currículo e a organização
	escolar, temas emergentes, estas temáticas repercutiram direta e indiretamente no
	desenvolvimento das práticas dos professores, e implicavam no processo de
	inclusão escolar, considerando estas constatações, o produto final da pesquisa está
	em fase de elaboração.
	
	\vspace{\onelineskip}
	
	\noindent
	\textbf{Palavras-chave}: Currículo e Inclusão. Práticas pedagógicas. Formação docente.
	
\end{document}