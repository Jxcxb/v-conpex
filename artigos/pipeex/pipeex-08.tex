\documentclass[article,12pt,onesidea,4paper,english,brazil]{abntex2}

\usepackage{lmodern, indentfirst, nomencl, color, graphicx, microtype, lipsum}			
\usepackage[T1]{fontenc}		
\usepackage[utf8]{inputenc}		

\setlrmarginsandblock{2cm}{2cm}{*}
\setulmarginsandblock{2cm}{2cm}{*}
\checkandfixthelayout

\setlength{\parindent}{1.3cm}
\setlength{\parskip}{0.2cm}

\SingleSpacing

\begin{document}
	
	\selectlanguage{brazil}
	
	\frenchspacing 
	
	\begin{center}
		\LARGE ISOLAMENTO E SELEÇÃO DE FUNGOS ENDOFÍTICOS ANTAGONISTAS DE \textit{COLLETOTRICUM SPP.}\footnote{Trabalho realizado dentro da área de biotecnologia, com fomento do IFRO/ARINT.}
		
		\normalsize
		Ana Paula Oliveira Rodrigues\footnote{Bolsista PIPEEX, ana.ifroea@gmail.com, \textit{Campus} Colorado do Oeste.} 
		Rafael Henrique Pereira dos Reis\footnote{Orientador no IFRO, rafael.reis@ifro.edu.br, \textit{Campus} Colorado do Oeste.} 
		Paula Cristina Santos Baptista\footnote{Orientadora no IPB, pbaptista@ipb.pt, Instituto Politécnico de Bragança.} 
	\end{center}
	
	\noindent Microrganismos endofíticos são aqueles que habitam o interior de plantas, sem necessariamente lhes causar danos. Existem alguns patógenos que são organismos endofíticos. 
	Sua presença causa desequilíbrio fisiológico à planta. Os benefícios
	obtidos pela planta hospedeira resultante das interações com os endófitos têm sido
	foco de diversos estudos, inclusive sobre a produção de metabólitos secundários
	que possuem propriedades de interesse (STROBEL \& DAISY, 2003). Dessa forma,
	o objetivo da pesquisa foi identificar fungos endofíticos que tenham capacidade de
	inibir fungos do complexo \textit{Colletotrichum spp.}, causador da gafa da oliveira. 
	A pesquisa foi realizada no Instituto Politécnico de Bragança, sob orientação da
	professora Dra. Paula Baptista e seus alunos de doutorado que prestaram
	assistência durante todo período de mobilidade realizada pelo Programa de
	Internacionalização da Pesquisa, Ensino e Extensão do IFRO, em 2016. Foram
	coletadas azeitonas em diferentes estágios madurais. Duas espécies foram
	utilizadas: a madural (suscetível à gafa) e a cobrançosa (moderadamente tolerante à
	gafa). Os frutos coletados foram levados a laboratório e esterilizados com etanol e
	água sanitária; na sequência, foram fragmentados e acondicionados em placas de
	petri contendo meio PDA (Potato Dextrose Agar), onde fungos endofíticos tinham
	capacidade de se desenvolver. Depois do crescimento dos endofíticos que foram
	isolados das azeitonas, foi realizada constantemente sua replicação para que se
	obtivesse sempre esporos puros e viáveis. Os esporos do endofítico foram diluídos
	em solução \textit{Tween} alcançando uma quantidade de 10$^6$/ml de esporos.
	Posteriormente, a solução com esporos do fungo foi aplicada sobre azeitonas
	aparentemente saudáveis e acondicionadas em pequenos potes. Com o passar dos
	dias, observou-se que as azeitonas que haviam sido submetidas a solução com
	esporos do endofítico tiveram o crescimento da gafa aparentemente controlado.
	Estudos mais profundos necessitam ser realizados para se constatar que realmente
	o fungo endofítico exerce efeito inibitório sobre a gafa da oliveira. O estudo de
	organismos endofíticos na região norte, pode se aplicar ao controle de doenças em
	culturas de importância econômica, como por exemplo a Antracnose, que também é
	causada por fungos do complexo \textit{Colletotrichum spp.}
	
	\vspace{\onelineskip}
	
	\noindent
	\textbf{Palavras-chave}: Fungos. Gafa. Isolamento.
	
\end{document}